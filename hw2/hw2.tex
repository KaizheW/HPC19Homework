\documentclass[12pt]{article}

%% FONTS
%% To get the default sans serif font in latex, uncomment following line:
 \renewcommand*\familydefault{\sfdefault}
%%
%% to get Arial font as the sans serif font, uncomment following line:
%% \renewcommand{\sfdefault}{phv} % phv is the Arial font
%%
%% to get Helvetica font as the sans serif font, uncomment following line:
% \usepackage{helvet}
\usepackage[small,bf,up]{caption}
\renewcommand{\captionfont}{\footnotesize}
\usepackage[left=1in,right=1in,top=1in,bottom=1in]{geometry}
\usepackage{graphics,epsfig,graphicx,float,subfigure,color}
\usepackage{amsmath,amssymb,amsbsy,amsfonts,amsthm}
\usepackage{url}
\usepackage{boxedminipage}
\usepackage[sf,bf,tiny]{titlesec}
 \usepackage[plainpages=false, colorlinks=true,
   citecolor=blue, filecolor=blue, linkcolor=blue,
   urlcolor=blue]{hyperref}
\usepackage{enumitem}
\usepackage{verbatim}
\usepackage{tikz,pgfplots}

\newcommand{\todo}[1]{\textcolor{red}{#1}}
% see documentation for titlesec package
% \titleformat{\section}{\large \sffamily \bfseries}
\titlelabel{\thetitle.\,\,\,}

\newcommand{\bs}{\boldsymbol}
\newcommand{\alert}[1]{\textcolor{red}{#1}}
\setlength{\emergencystretch}{20pt}

\begin{document}

\begin{center}
  \vspace*{-2cm}
{\small MATH-GA 2012.001 and CSCI-GA 2945.001, Georg Stadler \&
  Dhairya Malhotra (NYU Courant)}
\end{center}
\vspace*{.5cm}
\begin{center}
\large \textbf{%%
Spring 2019: Advanced Topics in Numerical Analysis: \\
High Performance Computing Assignment 2 \\
Kaizhe Wang (kw2223)}
\end{center}



\begin{enumerate}
% --------------------------
 \item {\bf Finding Memory bugs.}   \\
 Comments in the codes.

  \item {\bf Optimizing matrix-matrix multiplication.} \\
  The processor I used for computing is: \texttt{Intel(R) Core(TM) i5-7287U CPU @ 3.30GHz, 2 Cores, 4 Threads, Max Turbo Frequency 3.7GHz}. The block size I used is 32, and I found this block size has better performance than block size 16. \\
  The time for various matrix size obtained with the blocked version:
  \begin{center}
\begin{tabular}{ |c|c|c|c| } 
 \hline
 Dimension    &   Time   & Gflop/s    &   GB/s\\
 \hline
        32   &0.122775  &16.290187 &260.642984\\
       512   &0.106976 & 20.074443 &321.191093\\
        992  & 0.189864  &20.566121 &329.057932\\
       1472 &  0.324522  &19.656640 &314.506239\\
       1952  & 0.823768 & 18.057787 & 288.924587\\
  \hline
\end{tabular}
\end{center}

The time for various matrix size obtained with the blocked OpenMP version (number of thread is 4): 
  \begin{center}
\begin{tabular}{ |c|c|c|c| } 
 \hline
 Dimension    &   Time   & Gflop/s    &   GB/s\\
 \hline
         32  & 1.576696  & 1.268493 & 20.295886\\
       512  & 0.054245 & 39.588601& 633.417612\\
        992  & 0.099871  &39.098096 &625.569537\\
       1472  & 0.159663 & 39.952977 &639.247625\\
       1952   &0.384915 & 38.646004 &618.336072\\
  \hline
\end{tabular}
\end{center}

For size 32, OpenMP needs to initialize multi-threads, so it took longer than the unparalleled version. 

  
\item {\bf Finding OpenMP bugs.} \\
Comments in the codes.

\item {\bf OpenMP version of 2D Jacobi/Gauss-Seidel smoothing.}\\
The processor I used for computing is: \texttt{Intel(R) Xeon(R) CPU  E5630  @ 2.53GHz} (CIMS Compute Servers). The iteration number I used is 1000.\\
For the Jacobi method, the time for different matrix size and different numbers of threads are shown in the table below:
\begin{center}
\begin{tabular}{ |c|c|c|c|c|c| }
 \hline
 Matrix Size    &   1 Thread   & 2 Threads    &   4 Threads  & 8 Threads & 16 Threads\\
 \hline
         100  & 0.535856  & 0.274605 & 0.144738 & 0.082691 & 0.073158\\
         200  & 2.139034 & 1.077112 & 0.548631 & 0.294177 & 0.239523 \\
         500  & 13.344843 & 6.686164& 3.381802 & 1.773384 & 1.391263 \\
       1000  & 53.563724  & 26.618653 & 13.676684 & 7.086718 & 5.523454 \\
  \hline
\end{tabular}
\end{center}
We can see that if the number of threads is less than 8, the parallel version speed up the program significantly, but for 16 threads, there isn't much improvement from 8 threads. \\
For the Gauss-Seidel method, the time for different matrix size and different numbers of threads are shown in the table below:
\begin{center}
\begin{tabular}{ |c|c|c|c|c|c| }
 \hline
 Matrix Size & 1 Thread & 2 Threads & 4 Threads & 8 Threads & 16 Threads\\
 \hline
         100  & 0.545459  & 0.279640 & 0.152417 & 0.087271 & 0.082813\\
         200  & 2.173655 & 1.095316 & 0.561747 & 0.302996 & 0.251964 \\
         500  & 13.555463 & 6.777217 & 3.466832 & 1.802307 & 1.424178 \\
       1000  & 54.663060 & 27.137006 & 13.756388 & 7.184554 & 5.576750 \\
  \hline
\end{tabular}
\end{center}
  
  
\end{enumerate}


\end{document}
